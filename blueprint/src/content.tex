% In this file you should put the actual content of the blueprint.
% It will be used both by the web and the print version.
% It should *not* include the \begin{document}
%
% If you want to split the blueprint content into several files then
% the current file can be a simple sequence of \input. Otherwise It
% can start with a \section or \chapter for instance.




%

\begin{theorem}[Fundamental theorem on homomorphisms]
\lean{quotientKerEquivRange}

$G,G'$を群とし,$\phi:G\to G'$を群準同型とする.この時,
\begin{align}
    G/\ker\phi &\cong \phi(G) 
\end{align}
が成り立つ.
\end{theorem}
\begin{proof}
    $\phi$から誘導される写像$\tilde{\phi} : G/\ker\phi\to \phi(G)$を
    \begin{align}
        \tilde{\phi}(g\ker\phi) &= \phi(g) 
    \end{align}
    と定めたとき,これがwell-definedであり,かつ同型写像であることを示す.

    $\tilde{\phi}$がwell-definedであることを示す.
    $\ker\phi$の元を$k$とすると,$\phi(gk)=\phi(g)$が成り立つ.実際,$\phi$は群準同型であるから,
    \begin{align}
        \phi(gk) &= \phi(g)\phi(k) \notag\\
        &= \phi(g)e \notag\\
        &= \phi(g) \notag
    \end{align}
    が成り立つ.

    したがって,$\tilde{\phi}(g\ker\phi)=\phi(g)$が成り立つ.
    よって,$\tilde{\phi}$はwell-definedである.\\

    $\tilde{\phi}$は全射であることを示す.

    任意の$g'\in \phi(G)$に対して,$\phi(g)=g'$となる$g\in G$が存在する.
    よって,$\tilde{\phi}(g\ker\phi)=g'$が成り立つ.
    よって,$\tilde{\phi}$は全射である.
    
    
    $\tilde{\phi}$が単射であることを示す.

    $\tilde{\phi}(g_1\ker\phi)=\tilde{\phi}(g_2\ker\phi)$とすると,
    \begin{align}
        \phi(g_1) &= \phi(g_2)
    \end{align}
    が成り立つ.
    よって,$g_1^{-1}g_2\in \ker\phi$が成り立つ.
    すなわち,$g_1\ker\phi=g_2\ker\phi$が成り立つ.
    よって,$\tilde{\phi}$は単射である.

    以上より,$\tilde{\phi}$は全単射である.
    $\tilde{\phi}$は群準同型であることを示す.

    任意の$g_1\ker\phi,g_2\ker\phi\in G/\ker\phi$に対して,
    \begin{align}
        \tilde{\phi}(g_1\ker\phi g_2\ker\phi) &= \tilde{\phi}((g_1g_2)\ker\phi) \notag\\
        &= \phi(g_1g_2) \notag\\
        &= \phi(g_1)\phi(g_2) \notag\\
        &= \tilde{\phi}(g_1\ker\phi)\tilde{\phi}(g_2\ker\phi) \notag
    \end{align}
    が成り立つ.
    よって,$\tilde{\phi}$は群準同型である.
    $\tilde{\phi} : G/\ker\phi\to \phi(G)$は全単射かつ群準同型なので,$G/\ker\phi \cong \phi(G)$が成り立つ.
\end{proof}


\begin{theorem}[second isomorphism theorem]
\lean{secondIsomorphismTheorem}

$G$を群,$H \leq   G$を部分群,$N\lhd G$を正規部分群とする.この時,
\begin{align}
    H/(H\cap N) &\cong HN/N
\end{align}
が成り立つ.
\end{theorem}
\begin{proof}
    $\phi : H\to HN/N$を
    \begin{align}
        \phi(h) &= he_N N
    \end{align}
    と定める.この時,$\phi$は群準同型である.  

    $\ker\phi=H\cap N$が成り立つことを示す.

    $h \in \ker\phi$と仮定すると,$\phi(h)=eN$が成り立つ任意の$h\in H$に対して,$hN=N$が成り立つため$h\in N$が成り立つ.
    よって,$h\in H\cap N$が成り立つ.

    よって,$\ker\phi\subset H\cap N$が成り立つ.

    また任意の$h\in H\cap N$に対して,$\phi(h)=hN=eN$が成り立つため,$\ker\phi\supset H\cap N$が成り立つ.

    よって,$\ker\phi=H\cap N$が成り立つ.

    $\phi$が全射であることを示す.
    
    任意の$hnN\in HN/N$に対して,$h\in H,\phi(h)=hN=hnN$が存在するので,$\phi$は全射である.

    $\ker\phi=H\cap N$,$\phi(H)=HN/N$が成り立つので,Theorem1より
    \begin{align}
        H/(H\cap N) &\cong HN/N
    \end{align}
    が成り立つ.
\end{proof}
\begin{theorem}[third isomorphism theorem]
\lean{thirdIsomorphismTheorem}
$G$を群,$N\lhd G$,$M\lhd G$を正規部分群とする.$N\subseteq M$の時,
\begin{align}
    (G/N)/(M/N) &\cong G/M
\end{align}
が成り立つ.
\end{theorem}
\begin{proof}
    $\phi : G/N\to G/M$を
    \begin{align}
        \phi(gN) &= gM
    \end{align}
    と定める.この時,$\phi$は群準同型である.

    $\ker\phi=M/N$が成り立つことを示す.

    $\ker\phi = \left\{ gN \in G/N | \phi(gN) = M\right\}$より$gN\in \ker\phi$ならば$N \subseteq M$より$g\in M$.なので$\ker\phi \subset M/N$.

    また,$gN\in M/N$ならば$g\in M$より$\ker\phi \supset M/N$.よって$\ker\phi=M/N$,

    $\phi$が全射であることを示す.

    任意の$gM\in G/M$に対して,$N \subseteq M$より$gN\in G/N,\phi(gN)=gM$が存在するので,$\phi$は全射である.

    $\ker\phi=M/N$, $\phi(G/N)=G/M$が成り立つので,Theorem1より
    \begin{align}
        (G/N)/(M/N) &\cong G/M
    \end{align}
    が成り立つ.
\end{proof}
\begin{theorem}[fourth isomorphism theorem]
\lean{correspondence theorem}
$G$を群,$N\lhd G$を正規部分群とする.この時,$\cong $を順序同型として
    \begin{align}
        \left\{ K | K \leq  G/N\right\}\cong \left\{ H | H\leq  G , N \leq   H \right\}
    \end{align}
が成り立つ.つまり,$G/N$の部分群全体と$N$を含む$G$の部分群全体に対して,包含関係を保つ全単射が存在する.
%$G$を群,$N\lhd G$を正規部分群とする.この時,$G(N)$を$N$を含む$G$の部分群全体の集合とし,$A\to A/N$
%$A,B\in G(N)$に対して$A/N \leq B/N$を順序とすると,次の性質が成り立つ.
%    \begin{align}
%        A \leq B &\Leftrightarrow  A/N \leq B/N \\
%        A \leq B &\Rightarrow \left\lvert B : A\right\rvert = \left\lvert B/N : A/N\right\rvert \\
%        \left\langle A,B\right\rangle / N &= \left\langle A/N,B/N\right\rangle \\ 
%        \left(A\cap B\right) / N &= \left(A/N\cap B/N\right) \\
%        A \lhd G &\Leftrightarrow  A/N \lhd G/N
%    \end{align}
\end{theorem}
\begin{proof}

%    $f(K) := \left\{g\in G | gN\in K\right\} $と定義すると,
    \begin{align}
        X&:=\left\{ K | K \leq  G/N\right\} \\
        Y&:=\left\{ H | H\leq  G , N \leq   H \right\}
    \end{align}
    と定義する.また,以下の写像を定義する.
    \begin{align}
        \phi : Y \to X;H &\mapsto \left\{hN \in G/N | h \in H \geq  N\right\} \\
        \psi : X \to Y;K &\mapsto \left\{g\in G | gN \in K \leq  G/N \right\}
    \end{align}
    この時,$\phi$は全単射であることを示す.

    $\phi$が全射であることを示す.
    任意の$K\in X$に対して,$H=\left\{g\in G | gN \in K\right\}$が存在し,$\phi(H)=K$となる.この$H$は常に$N$を含むので,$H$は$N$を含む部分群である.
    また,$K\leq G/N$であるから,$H\leq G$が成り立つ.
    よって,$\phi$は全射である.

    $\psi$が全射であることを示す.
    任意の$H\in Y$に対して,$K=\left\{hN \in G/N | h \in H\right\}$が存在し,$\psi(K)=H$となる.$H$が$G$の部分群より$K$は$G/N$の部分群である.
    また,$N\leq H$であるから,$K\leq G/N$が成り立つ.
    よって,$\psi$は全射である.

    $\phi\circ\psi$が恒等写像であることを示す.
    任意の$K\in X$に対して,
    \begin{align}
        \phi(\psi(K)) &= \phi\left(\left\{g\in G | gN \in K\right\}\right) \notag\\
        &= \left\{hN \in G/N | h \in \left\{g\in G | gN \in K\right\}\right\} \notag\\
        &= \left\{gN \in G/N | gN \in K\right\} \notag\\
        &= K \notag
    \end{align}
    が成り立つ.
    よって,$\phi\circ\psi$は恒等写像である.

    $\psi\circ\phi$が恒等写像であることを示す.
    任意の$H\in Y$に対して,
    \begin{align}
        \psi(\phi(H)) &= \psi\left(\left\{hN \in G/N | h \in H\right\}\right) \notag\\   
        &= \left\{g\in G | gN \in \left\{hN \in G/N | h \in H\right\}\right\} \notag\\
        &= \left\{h \in G | h \in H\right\} \notag\\
        &= H \notag
    \end{align}
    が成り立つ.
    よって,$\psi\circ\phi$は恒等写像である.

    $\phi\circ\psi$と$\psi\circ\phi$が恒等写像であるので,$\phi$は全単射である.

    $H_1 \leq H_2$ならば$\phi(H_1) \leq \phi(H_2)$が成り立つことを示す.
    任意の$H_1,H_2\in Y$に対して,
    \begin{align}
        \phi(H_1) &= \left\{hN \in G/N | h \in H_1 \supset N\right\} \notag\\
        \phi(H_2) &= \left\{hN \in G/N | h \in H_2 \supset N\right\} \notag
    \end{align}
    が成り立つ.
    よって,$H_1 \leq H_2$ならば$H_1 \supset N$かつ$H_2 \supset N$が成り立つので,
    \begin{align}
        \phi(H_1) &= \left\{hN \in G/N | h \in H_1 \supset N\right\} \notag\\
        &\subseteq \left\{hN \in G/N | h \in H_2 \supset N\right\} \notag\\
        &= \phi(H_2) \notag
    \end{align}
    が成り立つ.
    よって,$\phi$は順序を保つ.

    $\phi : Y \to X$は全単射かつ順序を保つので,$X$と$Y$は順序同型である.

 
\end{proof}